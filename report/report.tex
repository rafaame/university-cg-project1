\documentclass[a4paper, 12pt]{article}

\usepackage[margin=1.0in]{geometry}
\usepackage[utf8]{inputenc}
\usepackage[brazil]{babel}
\usepackage{fancyhdr}
\usepackage{indentfirst}
\pagestyle{fancy}

\setlength{\headheight}{71.0pt}
\lhead
{
	Rafael Andreatta Martins (#USP 7564019)\\
	Rafael Silva de Milha (#USP 8139701)\\
  ICMC - USP - São Carlos - SP
}
\rhead{}

\begin{document}

\begin{center}
{\Large Projeto de Computação Gráfica (SCC-250)}
\end{center}

\vspace{.5cm}

\section{Controles}

É possível movimentar o personagem utilizando as setas do teclado ou as teclas W, S, A e D. O personagem também pode pular utilizando a barra de espaço.

\section{Compilando no Mac OS X}

Bibliotecas utilizadas (encontradas no diretório /vendor do projeto):

\begin{itemize}
  \item GLEW 1.11.0;
  \item GLFW 3.0.4;
  \item GLM 0.9.5.2 (não utilizada para linking);
  \item NVIDIA PhysX 3.3.2.
\end{itemize}
\vspace{.5cm}

Para facilitar a compilação um Makefile é disponibilizado no diretório raiz do projeto. Para compilar e rodar o projeto basta "make" ou "make run" no diretório raiz do projeto.

\end{document}